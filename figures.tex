% ============================================================
% Figures for the Berard Framework manuscript
% ============================================================

% ---------------------------
% FIGURE 1 — Stability Well + Oscillation
% ---------------------------
\begin{figure}[t]
    \centering
    \includegraphics[width=\linewidth]{figures/fig1_combined.pdf}
    \caption{
    (A) Quadratic stability well for the scalar field $\phi$, showing the curvature around the vacuum value $\phi_0$. The small-oscillation frequency is set by the effective mass $m_\phi = 2\pi f_0$ with $f_0 = 0.10\,\mathrm{Hz}$. 
    (B) Small-amplitude vacuum oscillation $\delta\phi(t)$ at the same frequency, illustrating the coherent mode that defines the resonance invariant $S_0$.
    }
    \label{fig:stability_well}
\end{figure}

% ---------------------------
% FIGURE 2 — Effective Mass Scaling
% ---------------------------
\begin{figure}[t]
    \centering
    \includegraphics[width=0.85\linewidth]{figures/fig2_effective_mass.pdf}
    \caption{
    Scaling of the effective mass $m_{\mathrm{eff}}/m$ as a function of the resonance invariant $S_0$. The relation $m_{\mathrm{eff}} = m\,BC^2 S_0^{-2}$ produces a steep sensitivity to small deviations around $S_0 = 1$, highlighted by the vertical reference line.
    }
    \label{fig:effective_mass}
\end{figure}

% ---------------------------
% FIGURE 3 — Hubble Parameter Rescaling
% ---------------------------
\begin{figure}[t]
    \centering
    \includegraphics[width=0.85\linewidth]{figures/fig3_hubble_rescaling.pdf}
    \caption{
    Rescaling of the Hubble parameter according to $H_B = H_{\mathrm{obs}}/BC$. The ratio $H_B/H_{\mathrm{obs}}$ decreases monotonically with increasing $BC$. The vertical line marks the fiducial value $BC = 1.054$ used throughout the analysis.
    }
    \label{fig:hubble_rescaling}
\end{figure}

% ---------------------------
% FIGURE 4 — Acceleration + Rotation Curves
% ---------------------------
\begin{figure}[t]
    \centering
    \includegraphics[width=\linewidth]{figures/fig4_combined.pdf}
    \caption{
    (A) Radial acceleration profiles for Newtonian gravity and the resonance-modified case $a = a_{\mathrm{Newton}}/BC^2$. The modification produces a uniform suppression across radii.
    (B) Corresponding rotation curves $v(r) = \sqrt{r\,a(r)}$. The resonance-modified curve is noticeably flatter than the Newtonian prediction, illustrating how the framework naturally generates flat galactic rotation curves without invoking dark matter.
    }
    \label{fig:rotation_curves}
\end{figure}

% ---------------------------
% FIGURE 5 — Framework Architecture
% ---------------------------
\begin{figure}[t]
    \centering
    \includegraphics[width=0.9\linewidth]{figures/fig5_framework_diagram.pdf}
    \caption{
    Conceptual flow diagram of the Berard Framework. The scalar field $\phi$ defines the resonance invariant $S_0$, which modifies the effective mass $m_{\mathrm{eff}}(\phi)$. This rescaling propagates to the effective energy density $\rho_{\mathrm{eff}} = \rho\,BC^2$, influencing both the inferred Hubble parameter $H_B$ and the large-scale acceleration $a_B$.
    }
    \label{fig:framework_diagram}
\end{figure}
